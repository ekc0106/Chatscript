\documentclass[]{article}
\usepackage{lmodern}
\usepackage{amssymb,amsmath}
\usepackage{ifxetex,ifluatex}
\usepackage{fixltx2e} % provides \textsubscript
\ifnum 0\ifxetex 1\fi\ifluatex 1\fi=0 % if pdftex
  \usepackage[T1]{fontenc}
  \usepackage[utf8]{inputenc}
\else % if luatex or xelatex
  \ifxetex
    \usepackage{mathspec}
  \else
    \usepackage{fontspec}
  \fi
  \defaultfontfeatures{Ligatures=TeX,Scale=MatchLowercase}
\fi
% use upquote if available, for straight quotes in verbatim environments
\IfFileExists{upquote.sty}{\usepackage{upquote}}{}
% use microtype if available
\IfFileExists{microtype.sty}{%
\usepackage[]{microtype}
\UseMicrotypeSet[protrusion]{basicmath} % disable protrusion for tt fonts
}{}
\PassOptionsToPackage{hyphens}{url} % url is loaded by hyperref
\usepackage[unicode=true]{hyperref}
\hypersetup{
            pdfborder={0 0 0},
            breaklinks=true}
\urlstyle{same}  % don't use monospace font for urls
\IfFileExists{parskip.sty}{%
\usepackage{parskip}
}{% else
\setlength{\parindent}{0pt}
\setlength{\parskip}{6pt plus 2pt minus 1pt}
}
\setlength{\emergencystretch}{3em}  % prevent overfull lines
\providecommand{\tightlist}{%
  \setlength{\itemsep}{0pt}\setlength{\parskip}{0pt}}
\setcounter{secnumdepth}{0}
% Redefines (sub)paragraphs to behave more like sections
\ifx\paragraph\undefined\else
\let\oldparagraph\paragraph
\renewcommand{\paragraph}[1]{\oldparagraph{#1}\mbox{}}
\fi
\ifx\subparagraph\undefined\else
\let\oldsubparagraph\subparagraph
\renewcommand{\subparagraph}[1]{\oldsubparagraph{#1}\mbox{}}
\fi

% set default figure placement to htbp
\makeatletter
\def\fps@figure{htbp}
\makeatother


\date{}

\begin{document}

\section{ChatScript Advanced Layers
Manual}\label{chatscript-advanced-layers-manual}

Copyright Bruce Wilcox, gowilcox@gmail.com www.brilligunderstanding.com
Revision 8/18/2019 cs9.62

\section{LAYERS}\label{layers}

ChatScript loads its startup data layers at a time.

Layer 0 comes from :build 0 typically, which is common stuff that comes
with ChatScript. You can override its contents by providing your own
files0.txt file. The layer is intended for stuff that won't be changing
much (thus saving recompilation time). Some people have used layer 0 to
house their entire bot, with layer 1 used for downloadable patches on
phone apps.

Layer 1 comes from :build harry or :build yourbot and is typically your
bot.

Layer 2 is a boot layer, which may not have any content. More later.

Layer 3 is the user layer. Every volley starts out by loading it from
the user's topic file. At the end of the volley data is written back out
to that file, and this layer is removed.

\subsection{BOOT layer}\label{boot-layer}

The boot layer is used for adding dynamic data to the basic server on
startup (since layers 0 and 1 are precompiled).

\subsubsection{\texorpdfstring{\texttt{\^{}CSBOOT()}}{\^{}CSBOOT()}}\label{csboot}

outputmacro: \^{}CSBOOT()

This function, if defined by you, will be executed on startup of the
ChatScript system. It is a way to dynamically add facts and user
variables into the base system common to all users (the boot layer). And
returned output will go to the console and if a server, into the server
log Note that when a user alters a system \texttt{\$variable}, it will
be refreshed back to its original value for each user.

If you create JSON data, you should probably use \^{}jsonlabel() to
create unique names separate from the normal json naming space.

Changing the content of boot json facts (e.g.
\texttt{\$data.value\ +=\ 1}) may create abandoned data in the boot
layer (the old value of \$data.value) and do this often enough and you
may run out of memory since there is no way to reclaim it.

All permanent facts created during the boot process will reside in this
layer while the server remains operational. If you restart the server,
whatever was stored here is gone, presumably recreated by a new call to
\^{}CSBOOT.

HOW DOES GC WORK HERE if you create transient facts?

You can also add facts to the boot layer from execution of user scripts.
Just create facts from a user script using \^{}createFact(s v o
\#FACTBOOT) and when layer 1 is being removed at the end of the volley,
these facts will migrate to the boot layer.

You may define any number of \^{}CSBOOT functions. This allows both
multibot and standalone environments to compile without triggering a
complaint about already defined functions, though only the first will
ever execute.

\subsubsection{\^{}purgeboot (what)}\label{purgeboot-what}

If you created facts previously during user script executing (via
CreateFact with a flags argument FACTBOOT) Or via fact creation in
\^{}csboot, then they would reside there unchanged as part of a server
until it goes down. Those facts are visible to all users and becaue they
are system owned facts, cannot be modified by a user. However, you can
use this function to remove them from the boot layer.

\begin{verbatim}
^purgeboot(@1)
\end{verbatim}

All facts you may have queried into this factset which are facts from
the boot layer will be removed.

\begin{verbatim}
^purgeboot(botid)
\end{verbatim}

All facts with the given numeric bot id in the boot layer will be
removed.

\subsubsection{\texorpdfstring{\texttt{\^{}CS\_REBOOT()}}{\^{}CS\_REBOOT()}}\label{cs_reboot}

\begin{verbatim}
outputmacro: ^CS_REBOOT()
\end{verbatim}

This function, if defined by you, will be executed on every volley prior
to loading user data. It is executed as a user-level program which means
when it has completed all newly created facts and variables just
disappear. Typically you would have the \^{}CS\_REBOOT function test
some condition (results of a version stamp) and if the version currently
loaded in the boot layer is current, it simply returns having done
nothing. If the boot layer is not current, then you call the cs function
\^{}REBOOT() which erases the current boot data and treats the remainder
of the script as refilling the boot layer with new facts and variables.

\subsubsection{Command Line Parameters}\label{command-line-parameters}

\paragraph{bootcmd=xxx}\label{bootcmdxxx}

runs this command string before CSBOOT is run. Use it to trace the boot
process or whatever.

\paragraph{recordboot}\label{recordboot}

Will append to the toplevel file bootfacts.txt file any data moved from
user layer to boot layer due to FACTBOOT facts being created. Normally
facts moved to the boot layer from user layer would just be lost when
the server restarts. This allows you to recover them, using
\^{}importfact to read back these facts on server startup.

\end{document}
